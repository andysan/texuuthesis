% \iffalse meta-comment
%
% Copyright (c) 2014 Andreas Sandberg
% -----------------------------------
%
% Redistribution and use in source and binary forms, with or without
% modification, are permitted provided that the following conditions are
% met: redistributions of source code must retain the above copyright
% notice, this list of conditions and the following disclaimer;
% redistributions in binary form must reproduce the above copyright
% notice, this list of conditions and the following disclaimer in the
% documentation and/or other materials provided with the distribution;
% neither the name of the copyright holders nor the names of its
% contributors may be used to endorse or promote products derived from
% this software without specific prior written permission.
%
% THIS SOFTWARE IS PROVIDED BY THE COPYRIGHT HOLDERS AND CONTRIBUTORS
% "AS IS" AND ANY EXPRESS OR IMPLIED WARRANTIES, INCLUDING, BUT NOT
% LIMITED TO, THE IMPLIED WARRANTIES OF MERCHANTABILITY AND FITNESS FOR
% A PARTICULAR PURPOSE ARE DISCLAIMED. IN NO EVENT SHALL THE COPYRIGHT
% OWNER OR CONTRIBUTORS BE LIABLE FOR ANY DIRECT, INDIRECT, INCIDENTAL,
% SPECIAL, EXEMPLARY, OR CONSEQUENTIAL DAMAGES (INCLUDING, BUT NOT
% LIMITED TO, PROCUREMENT OF SUBSTITUTE GOODS OR SERVICES; LOSS OF USE,
% DATA, OR PROFITS; OR BUSINESS INTERRUPTION) HOWEVER CAUSED AND ON ANY
% THEORY OF LIABILITY, WHETHER IN CONTRACT, STRICT LIABILITY, OR TORT
% (INCLUDING NEGLIGENCE OR OTHERWISE) ARISING IN ANY WAY OUT OF THE USE
% OF THIS SOFTWARE, EVEN IF ADVISED OF THE POSSIBILITY OF SUCH DAMAGE.
%
% \fi
%
% \iffalse
%
%<class>\NeedsTeXFormat{LaTeX2e}
%<class>\ProvidesClass{uuthesis}
%<class>  [2014/03/17 v0.1 Unofficial class file for Uppsala University Theses]
%
%<berling|gill|gillbook|gillheavy|gilllight>\ProvidesFile%
%<berling>{berling.fontspec}[2014/03/17 v0.1 UU Thesis Berling font specification]
%<gill>{gillaltonemt.fontspec}[2014/03/17 v0.1 UU Thesis Gill Sans font specification]
%<gillbook>{gillaltonemtbook.fontspec}[2014/03/17 v0.1 UU Thesis Gill Sans font specification]
%<gillheavy>{gillaltonemtheavy.fontspec}[2014/03/17 v0.1 UU Thesis Gill Sans font specification]
%<gilllight>{gillaltonemtlight.fontspec}[2014/03/17 v0.1 UU Thesis Gill Sans font specification]
%
%<*driver>
\documentclass{ltxdoc}
\usepackage[T1]{fontenc}
\usepackage{lmodern}
\usepackage[numbered]{hypdoc}
\EnableCrossrefs
\CodelineIndex
\RecordChanges

\begin{document}
\ifx\LuaTeX\undefined
\newcommand{\LuaLaTeX}{Lua\LaTeX}
\newcommand{\LuaTeX}{Lua\TeX}
\fi

\newcommand{\DescribeOpt}{\DescribeMacro}
\DocInput{\jobname.dtx}
\PrintChanges
\PrintIndex
\end{document}
%</driver>
%
% \fi
%
% \CheckSum{0}
%
%^^A ------------------------------------------------------------
%^^A Change log
%^^A ------------------------------------------------------------
% \changes{v0.1}{2014/03/17}{First public release}
%
%
%^^A ------------------------------------------------------------
%^^A Index overrides
%^^A ------------------------------------------------------------
%
% \DoNotIndex{\ClassError, \ClassWarning}
% \DoNotIndex{\CurrentOption, \DeclareOption, \PassOptionsToClass}
% \DoNotIndex{\LoadClass, \RequirePackage}
% \DoNotIndex{\AtBeginDocument}
%
% \DoNotIndex{\,, \\, \", \{, \} }
% \DoNotIndex{\begin, \end, \begingroup, \endgroup}
% \DoNotIndex{\if, \ifx, \else, \fi, \newif, \notblank}
% \DoNotIndex{\undefined}
% \DoNotIndex{\def, \let}
% \DoNotIndex{\g@addto@macro}
% \DoNotIndex{\newcommand, \renewcommand, \newenvironment, \renewenvironment}
% \DoNotIndex{\addtocounter, \newcounter, \refstepcounter, \stepcounter}
% \DoNotIndex{\thechapter, \thefigure, \thepage, \thesection, \thetable}
% \DoNotIndex{\arabic, \Roman, \roman}
% \DoNotIndex{\addtolength}
% \DoNotIndex{\makebox}
% \DoNotIndex{\MakeUppercase}
% \DoNotIndex{\par}
% \DoNotIndex{\relax}
% \DoNotIndex{\textwidth}
%
% \DoNotIndex{\directlua, }
%
% \DoNotIndex{\noindent, \flushright, \raggedright, \endflushright}
% \DoNotIndex{\hspace, \vfill, \vskip, \vspace}
% \DoNotIndex{\enumerate, \endenumerate, \item, \labelenumi}
%
% \DoNotIndex{\chapter, \section, \subsection, \subsubsection, \paragraph}
%
% \DoNotIndex{\cite, \fullcite, \newrefsection}
%
% \DoNotIndex{\defaultfontfeatures}
% \DoNotIndex{\@setfontsize, \setmainfont, \setsansfont}
% \DoNotIndex{\fontsec, \normalfont}
% \DoNotIndex{\upshape, \itshape, \slshape, \scshape}
% \DoNotIndex{\bfseries, \mdseries}
% \DoNotIndex{\sffamily, \rmfamily, \ttfamily}
% \DoNotIndex{\textmd, \textsf}
% \DoNotIndex{\tiny, \scriptsize, \footnotesize, \small, \normalsize}
% \DoNotIndex{\large, \Large, \LARGE, \huge, \Huge, \HUGE}
% \DoNotIndex{\baselineskip}
%
% \DoNotIndex{\checkandfixthelayout}
% \DoNotIndex{\setlrmargins, \setmarginnotes, \settrims, \settrimmedsize}
% \DoNotIndex{\settypeblocksize, \setulmargins}
% \DoNotIndex{\fussy, \midsloppy, \sloppy}
% \DoNotIndex{\quarkmarks}
%
% \DoNotIndex{\cleartorecto, \cleartoverso, \newpage}
% \DoNotIndex{\openright}
%
% \DoNotIndex{\headstyles, \makeevenfoot, \makeevenhead, \makeheadstyles}
% \DoNotIndex{\aliaspagestyle, \makepagestyle, \pagestyle, \thispagestyle}
% \DoNotIndex{\chapterstyle, \makechapterstyle}
% \DoNotIndex{\chapterheadstart, \chaptername, \chapternamenum}
% \DoNotIndex{\chapnamefont, \chapnumfont, \chaptitlefont}
% \DoNotIndex{\afterchapskip, \afterchapternum, \beforechapskip, \midchapskip}
%
% \DoNotIndex{\captionsswedish}
%
%^^A ------------------------------------------------------------
%^^A Documentation Start
%^^A ------------------------------------------------------------
%
% \title{An Unofficial Uppsala University Thesis Class}
% \author{^^A
%   Andreas Sandberg\\ \texttt{andreas@sandberg.pp.se}^^A
% }
% \date{}
% \maketitle
%
% \tableofcontents
%
% \newpage
% \section{Introduction}
% This is the unofficial Uppsala University thesis class for
% \LaTeX. Unlike the official class, this class builds on the
% excellent |memoir| class. The main purpose of this class is to
% provide a non-intrusive layer around memoir to setup the same (or
% sometimes a similar) layout as in the official template.
%
% In addition to the features provided by |memoir|, this class
% provides some helper environment that simplifies tasks related to
% theses writing. While the main focus has been on so called
% comprehensive summaries, the class should be usable for monographs
% as well.
%
% \section{Usage}
%
% Load the class just like any other \LaTeX{} class. That is, put the
% following in your preamble:
% \begin{quote}
%   |\documentclass[<|\textit{options}|>]{uuthesis}|
% \end{quote}
%
% \section{Class Options}
% \subsection{Style Options}
% The following options describe how the thesis is rendered. They
% control features such as layout and font selection. The default
% layout and font selection tries to adhere to the official thesis
% template.
%
%
% \DescribeOpt{altfonts}%
% Use Uppsala University's profile fonts (Berling \& Gill Sans). These
% need to be installed separately.
%
% \DescribeOpt{air}%
% Use a non-standard layout with bigger baseline skips.
%
% \subsection{Production Options}
% The following options control features that are useful when creating
% drafts and pre-prints of a thesis using a normal office printer.
%
% \DescribeOpt{trim}%
% \DescribeMacro{\revisioninfo}%
% Trim the paper to S5. Activating this option disables trim marks and
% sets up a crop box around the resulting S5 page to cause PDF readers
% to display the crop the generated A4 pages to S5. The resulting PDF
% will contain the same bleed box and trim box irregardless of this
% option. Users wishing to print revision information (e.g., from git)
% should overload the |\revisioninfo| macro whose contents is
% displayed in the lower right corner of the sheet.
%
% \DescribeOpt{final}%
% Configure the class for print output. This disables debug features
% such as overfull hbox rules. The option will be passed to |memoir|.
%
% \DescribeOpt{draft}%
% Setup features usable for draft production. Currently disables
% trimming and causes overfull rules to be display.
%
%
% \section{Environments}
%
% \DescribeEnv{dedication}%
% Typeset a dedication page. The text within the environment will be
% set in italics.
%
% \DescribeEnv{placeholderpage}%
% Create a placeholder page with the text within the environment.
%
%
% \section{Bibliographies}
% This class uses the |biblatex| package to typeset
% bibliographies. This provides some functionality that allows
% separate bibliographies in different papers in a comprehensive
% summary. See the |biblatex| manual for more information.
%
% \section{Typesetting Paper Lists}
% \DescribeEnv{paperlist}%
% \DescribeEnv{paperlist*}%
% \DescribeMacro{paper}%
% A comprehensive summary normally contains one or more lists of
% papers. This class contains provides two environments to typeset
% such a list. The |paperlist| environment produces an ordered list
% with roman numerals while the |paperlist*| environment produces an
% unordered list. Individual papers are typset using the |\paper|
% macro is defined within the environments. Calling this macro causes
% a bibliography entry to be printed. The macro takes cite key
% and an optional participation as its arguments: \\
% |\paper|~\marg{cite~key}~\oarg{participation}
%
% \subsection{Typesetting Papers in Comprehensive Summaries}
% \DescribeEnv{abstract}%
% The abstract environment is primarily indented to be used when
% typesetting papers. However, it is frequently used to typeset the
% abstract placeholder page.
%
% \DescribeMacro{\papers}%
% At the end of the comprehensive summary, the |\papers| command is
% use to shift to paper production.
%
% \DescribeMacro{\paper}%
% Once shifted to paper production mode with the |\papers| command, a
% paper heading can be created using the |\paper| macro. This macro
% should be called as follows:\\
% |\paper|~\marg{title}~\marg{authors}~\marg{copyright}
%
% \subsection{Using Fonts}
% The class will automatically create macros to switch to fonts from
% the University's typographic profile if the class option |altfonts|
% has been specified.
%
% \DescribeMacro{\berling}%
% This is an official font from the typographic profile of the
% university. Its prescribed use is in the body of a larger text.
%
% \DescribeMacro{\gill}%
% This macro switches ``Gill Alt One MT''. This is the official sans
% serif font from the typographic profile of the university. Its
% prescribed use is in figure headings, texts, short fact boxes,
% diagrams, and pagination.
%
% \DescribeMacro{\gillheavy}%
% This macro switches ``Gill Alt One MT Heavy''. This is not a part of the
% official typographic profile.
%
% \DescribeMacro{\gilllight}%
% This macro switches ``Gill Alt One MT Light''. This is not a part of
% the official typographic profile.
%
% \DescribeMacro{\gillbook}%
% This macro switches ``Gill Alt One MT Book''. This is not a part of
% the official typographic profile. However, it matches Berling nicely
% and might be a better choice for ``highlighting'' text in a body of
% text set in Berling as Gill Alt One MT might stand out too much.
%
%
% \subsection{Installing Fonts}
% Download the university's font package from the university
% website. Make sure to download the PC version. The file should be
% called something like |18350_TypeSnitt_PC.zip|.
%
% Copy the directories |Berling PS| and |Gill PS| to your thesis
% directory. \LuaLaTeX{} should now be able to find the font files.
%
% \section{Implementation}
%\iffalse
%<*class>      
% \fi
%
% \subsection{Option Handling}
%  \begin{macro}{\ifuut@overfullrule}
%    Highlight overfull hboxes. Default:~|true|.
%    \begin{macrocode}
\newif\ifuut@overfullrule\uut@overfullruletrue
%    \end{macrocode}
% \end{macro}
% 
% \begin{macro}{\ifuut@trimmed}
% \begin{macro}{\iftrimmed}
%   Trim page (show trim marks otherwise). This is exported as
%   |\iftrimmed|. Default:~|false|.
%    \begin{macrocode}
\newif\ifuut@trimmed\uut@trimmedfalse
\def\iftrimmed{\ifuut@trimmed}
%    \end{macrocode}
% \end{macro}
% \end{macro}
% 
% \begin{macro}{\ifuut@altfonts}
%   Use CMR instead of Berling/Gill Sans. Default:~|false|
%    \begin{macrocode}
\newif\ifuut@altfonts\uut@altfontsfalse
%    \end{macrocode}
% \end{macro}
%
% \begin{macro}{\ifuut@air}
%   Use a non-standard layout that does not look as crowded as the
%   official layout. Default:~|false|
%    \begin{macrocode}
\newif\ifuut@air\uut@airfalse
%    \end{macrocode}
% \end{macro}
%
% \begin{macro}{\ifuut@sloppybottom}
%   Enable a sloppy bottom in memoir and set higher penalties for
%   widows and orphans. This also adjusts the margins to prevent the
%   text block from hitting the pagination when additional lines are
%   crammed into the text block to prevent a
%   widow/orphan. Default:~|false|
%    \begin{macrocode}
\newif\ifuut@sloppybottom\uut@sloppybottomfalse
%    \end{macrocode}
% \end{macro}
%
%    \begin{macrocode}
\DeclareOption{final}{
  \PassOptionsToClass{\CurrentOption}{memoir}
  \uut@overfullrulefalse
}

\DeclareOption{draft}{
  \PassOptionsToClass{\CurrentOption}{memoir}
  \uut@overfullruletrue
  \uut@trimmedfalse
}

\DeclareOption{trim}{
  \uut@trimmedtrue
}

\DeclareOption{altfonts}{
  \uut@altfontstrue
}

\DeclareOption{air}{
  \uut@airtrue
}

\DeclareOption{sloppybottom}{
  \uut@sloppybottomtrue
}

\DeclareOption*{
  \ClassWarning{uuthesis}{Unknown option '\CurrentOption'}
}
%    \end{macrocode}
%
% Process the options and setup memoir.
%    \begin{macrocode}
\ProcessOptions\relax

\iftrimmed
\else
  \PassOptionsToClass{showtrims}{memoir}
\fi

\LoadClass[11pt,twoside,onecolumn]{memoir}
%    \end{macrocode}
%
% \begin{macro}{\ifuut@lualatex}
%   We need to check for \LuaLaTeX{} for some features, create a macro
%   that does that.
%    \begin{macrocode}
\newif\ifuut@lualatex
\ifx\directlua\uut@undefined
\uut@lualatexfalse
\else
\uut@lualatextrue
\fi
%    \end{macrocode}
% \end{macro}
%
% Check that options are sane and load additional packages.
% 
%    \begin{macrocode}
%      
\RequirePackage{graphics}
\ifuut@lualatex
\RequirePackage{fontspec}
\RequirePackage{luatexbase,luacode}
\fi

% Switch to CMR if we aren't using lualatex (i.e., can't use fontspec)
\ifuut@lualatex
\else
\ifuut@altfonts
\ClassError{uuthesis}{%
  Can not use alternative fonts unless LuaLaTeX is used.%
}{%
  Render the document using lualatex.
}%
\fi
\uut@altfontsfalse
\fi
%    \end{macrocode}
%
% \subsection{Setup Page Layout}
%
% Override baseline skip to get something that resembles the official
% class.
%    \begin{macrocode}
% 
\ifuut@air
\newcommand{\uut@baselineskip@normal}{13.2pt}
\else
\newcommand{\uut@baselineskip@normal}{13.0pt}
\fi

\let\uut@old@normalsize\normalsize%
\renewcommand\normalsize{%
  \uut@old@normalsize%
  \@setfontsize{\normalsize}{11pt}{\uut@baselineskip@normal}%
}%
\normalsize
%    \end{macrocode}
%
% Setup paper size. We only support S5 printed on A4 at the moment.
%
%    \begin{macrocode}
\stockaiv % 210x297 (595.28bp x 841.89bp)
\settrims{27.5mm}{22.5mm}
\settrimmedsize{242mm}{165mm}{*}
%    \end{macrocode}
%
% Setup type block and margins. The official typeblock is normally
% 572pt high, but we adjust it slightly since we use 11pt topskip
% instead of 13pt to maintain a 43 line type block. We need to
% decrease the top margin slightly and move the type block when using
% the sloppy bottom feature. This prevents sloppy bottom overflows
% from ``hitting'' the pagination (causing major ugliness in the
% process).
%    \begin{macrocode}
\ifuut@air
\settypeblocksize{565.4pt}{120mm}{*}
\else
\settypeblocksize{570pt}{120mm}{*}
\fi
\setlrmargins{22.5mm}{*}{*}
\setulmargins{18mm}{*}{*}
\setheaderspaces{*}{\onelineskip}{*}
\setheadfoot{\onelineskip}{2\onelineskip}
\setmarginnotes{1mm}{20mm}{5pt}
\checkandfixthelayout
%    \end{macrocode}
%
% Prevent orphans and widows by making their penalty really high and
% enable the sloppy bottom feature of the memoir class.
%    \begin{macrocode}
\ifuut@sloppybottom
\clubpenalty=9996
\widowpenalty=9999
\brokenpenalty=4991

\sloppybottom
\fi
%    \end{macrocode}

%
% Setup PDF metadata to let the print shop know where to trim the page
% (|TrimBox|) and how much bleed to print (|BleedBox|). Set the
% |CropBox| as well to make the PDF reader only display the trimmed
% page if the |trimmed| option has been passed to the class.
%
%    \begin{macrocode}
\iftrimmed%
\newcommand{\uut@setuppdfbox}{
  \pdfpageattr{
    /MediaBox [0 0 595.28 841.89]
    /BleedBox [51.78 65.955 543.50 775.93]
    /TrimBox [63.78 77.955 531.50 763.93]
    /CropBox [63.78 77.955 531.50 763.93]
  }%
}
\else%
\newcommand{\uut@setuppdfbox}{
  \pdfpageattr{
    /MediaBox [0 0 595.28 841.89]
    /BleedBox [51.78 65.955 543.50 775.93]
    /TrimBox [63.78 77.955 531.50 763.93]
  }%
}
\fi%

\AtBeginDocument{%
  \uut@setuppdfbox%
}%
%    \end{macrocode}
%
% \subsection{Trim marks}
% \begin{macro}{revisioninfo}
% \begin{macro}{uutrimmarks}
%   Setup trimmarks based on |\quarkmarks|, but add a hook to display
%   revision control information at the bottom and display the sheet
%   number at the top.
%    \begin{macrocode}
\newcommand{\revisioninfo}{}

\newcommand{\uuttrimmarks}{%
  \quarkmarks

  \g@addto@macro\tmarktr{%
    \begin{picture}(0,0)%
      \put(-3,27){%
        \makebox[0pt][r]{
          \normalfont\ttfamily\fontsize{8bp}{10bp}\selectfont%
          Sheet: \thesheetsequence\ of \thelastsheet%
        }%
      }%
    \end{picture}%
  }

  \g@addto@macro\tmarkbl{%
    \begin{picture}(0,0)%
      \put(0,-50){%
        \normalfont\ttfamily\fontsize{8bp}{10bp}\selectfont%
        \revisioninfo%
      }%
    \end{picture}%
  }
}
%    \end{macrocode}
% \end{macro}
% \end{macro}
%
% Enable trim marks by default when when not trimmed.
%    \begin{macrocode}
\iftrimmed
\else
\uuttrimmarks
\fi
%    \end{macrocode}
%
% \subsection{Setup Fonts}
% \subsubsection{Alternative Style}
%    \begin{macrocode}
\ifuut@altfonts
%    \end{macrocode}
%
%   Setup font macros for sections and chapters based on the font
%   requested. Setup font aliases and ligatures if alternative fonts
%   have been requested.
%
%   First, setup aliases for the fonts from the typographic profile
%   and enable TeX ligatures by default.
%    \begin{macrocode}
\defaultfontfeatures{
  Ligatures = {TeX},
}

\newfontfamily\berling{Berling}
\newfontfamily\gill{GillAltOneMT}
\newfontfamily\gillheavy{GillAltOneMTHeavy}
\newfontfamily\gilllight{GillAltOneMTLight}
\newfontfamily\gillbook{GillAltOneMTBook}
%    \end{macrocode}
%
%    \begin{macrocode}
\setmainfont{Berling}
\setsansfont{GillAltOneMT}
%    \end{macrocode}
%
% \begin{macro}{\uut@font@sec}
% \begin{macro}{\uut@font@chapter}
%   Second, define aliases for the fonts we are going to use in
%   sections and chapters,.
%   \begin{macrocode}
\newcommand{\uut@font@abstract}{\gill\bfseries}
\newcommand{\uut@font@chapter}{\gill}
\newcommand{\uut@font@pagination}{\gillbook\mdseries\upshape}
\newcommand{\uut@font@paper}{\gill\huge}
\newcommand{\uut@font@paperno}{\gill\HUGE}
\newcommand{\uut@font@sec}{\gill}
%    \end{macrocode}
% \end{macro}
% \end{macro}
%
% Use Gill Alt One MT for captions in floats (e.g., ``Figure'') and
% Gill Alt One MT Book for the actual text in the caption. The
% original Gill font looks a bit too dark in contrast with the body
% text in Berling, but it makes the caption name stand out nicely.
%    \begin{macrocode}
\captionnamefont{\small\gill\mdseries\upshape}
\captiontitlefont{\small\gillbook\mdseries\upshape}
\captionstyle{}
\captiondelim{: }

\subcaptionsize{}
\subcaptionlabelfont{\small\gill\mdseries\upshape}
\subcaptionfont{\small\gillbook\mdseries\upshape}
%    \end{macrocode}
%
% \subsubsection{Default Style}
%    \begin{macrocode}
\else
%    \end{macrocode}
%
% \begin{macro}{\uut@font@sec}
% \begin{macro}{\uut@font@chapter}
%    \begin{macrocode}
\newcommand{\uut@font@abstract}{\sffamily\bfseries}
\newcommand{\uut@font@chapter}{\sffamily\bfseries}
\newcommand{\uut@font@pagination}{\rmfamily\mdseries\upshape}
\newcommand{\uut@font@paper}{\sffamily\huge\bfseries}
\newcommand{\uut@font@paperno}{\sffamily\HUGE}
\newcommand{\uut@font@sec}{\bfseries\sffamily}
%    \end{macrocode}
% \end{macro}
% \end{macro}
%
%    \begin{macrocode}
\fi
%    \end{macrocode}
%
% \subsection{Draft Debug}
% 
% \subsubsection{Overfull hbox printing}
% The following code sets up printing of overfull hbox margin
% markers. The main point of this code is to highlight them in color
% when using \LuaLaTeX.
%
%    \begin{macrocode}
\ifuut@overfullrule
\overfullrule 5pt

\ifuut@lualatex
% Kudos to topskip on tex.stackexchange.com.
\begin{luacode}
magentabox = function(head)
  while head do
    if head.id == 0 or head.id == 1 then
      -- go through the hlists (the rows)
      magentabox(head.head)

    -- if there's a rule after the rightskip, this is the overfull box
    -- node id 10 == glue, glue subtype 9 is rightskip, node id 2 is a rule

    elseif head.id == 10 and head.subtype == 9 and head.next and head.next.id == 2 then
       -- this must be an overfull box
       local w1, w2
       w1 = node.new("whatsit","pdf_literal")
       w1.data = "q 1 0 1 rg"
       w1.mode = 1
       w2 = node.new("whatsit","pdf_literal")
       w2.data = " Q"
       w2.mode = 1

       w1.next = head.next -- the rule
       head.next = w1      -- color start
       w1.next.next = w2   -- color end

       node.slide(head)    -- adjust prev pointers
    end
    head = head.next
  end
  return true
end
luatexbase.add_to_callback("post_linebreak_filter",magentabox,"magentabox")
\end{luacode}
\fi %lualatex

\fi %uut@overfullrule
%    \end{macrocode}
% 
% \subsection{Page Style}
%
% \begin{macro}{\uut@addsuffix}
%   Helper macro that concatenates two parameters. This can be used to
%   append a string to the next to a parameter following the
%   macro invocation.
% \begin{macrocode}
\newcommand{\uut@addsuffix}[2]{#2#1}
% \end{macrocode}
% \end{macro}
% 
% \subsubsection{Main Page Style}
% \begin{macro}{\ps@uut}
%   Declare the default thesis page style.
%    \begin{macrocode}
\makepagestyle{uut} \makeevenfoot{uut}{{\uut@font@pagination\thepage}}{}{}
\makeoddfoot{uut}{}{}{{\uut@font@pagination\thepage}}
%    \end{macrocode}

% \end{macro}
%    % 
% Setup aliases for the page style to ensure that it gets used
% consistently throughout the thesis.
%    \begin{macrocode}
\aliaspagestyle{part}{uut}
\aliaspagestyle{book}{uut}
\aliaspagestyle{chapter}{uut}
%    \end{macrocode}
%
% 
% \begin{macro}{\chs@uut}
%   Create a default chapter style that resembles the official chapter
%   style.
%    \begin{macrocode}
\makechapterstyle{uut}{%
  \chapterstyle{section}
  \renewcommand*{\chaptitlefont}{\uut@font@chapter\Huge}
  \renewcommand*{\chapnumfont}{\chaptitlefont}
  \renewcommand*{\printchaptertitle}[1]{\chaptitlefont ##1}
  \renewcommand{\afterchapternum}{\hspace{12pt}}
  \setlength{\beforechapskip}{-\baselineskip}
  \setlength{\afterchapskip}{80pt}
}
%    \end{macrocode}
% \end{macro}
%
% \begin{macro}{\hds@uut}
%    \begin{macrocode}
\newcommand{\uut@secstyle}{\uut@font@sec\raggedright}
\makeheadstyles{uut}{%
  \setsecheadstyle{\fontsec\Large}%
  \setsubsecheadstyle{\fontsec\large}%
  %
  % chapter
  \chapterstyle{uut}%
  %
  % section
  \setsecheadstyle{\uut@secstyle\Large}%
  \setbeforesecskip{-4.0ex \@plus -3ex \@minus -0.7ex}
  \setaftersecskip{2.3ex \@plus 0.2ex}
  %
  % subsection
  \setsubsecheadstyle{\uut@secstyle\large}%
  \setbeforesubsecskip{-3.5ex \@plus -1.5ex \@minus -0.5ex}
  \setaftersubsecskip{1.5ex \@plus 0.5ex}
  %
  % subsubsection
  \setsubsubsecheadstyle{\uut@secstyle\normalsize}%
  \setbeforesubsubsecskip{-3.5ex \@plus -1.5ex \@minus -0.5ex}
  \setaftersubsubsecskip{1.5ex \@plus 0.5ex}
  %
  % paragraph
  \setparaheadstyle{\normalfont\normalsize\bfseries\uut@addsuffix{:
    }}%
  \setparaindent{\parindent}%
  \setbeforeparaskip{0ex}%
  \setafterparaskip{0em}%
}%
%    \end{macrocode}
% \end{macro}
%
% Enable the thesis page style and header style by default.
%    \begin{macrocode}
\pagestyle{uut}
\headstyles{uut}
%    \end{macrocode}
%
% Allow more stretch before and after figures and captions. This makes
% it easier for \TeX{} to avoid ugly page breaks.
%    \begin{macrocode}
\setlength{\abovecaptionskip}{%
  0.5\onelineskip \@plus 0.5\onelineskip}
\setlength{\belowcaptionskip}{%
  0.5\onelineskip \@plus 2.5\onelineskip \@minus 0.5\onelineskip}
\setlength{\subfloattopskip}{5pt \@plus 5pt}
\setlength{\subfloatbottomskip}{5pt \@plus 5pt}
\setlength{\subfloatcapskip}{0pt \@plus 2.5pt}
%    \end{macrocode}
%
% \subsubsection{Placeholder Pages}
% Define the placeholder page style.
%    \begin{macrocode}
\def\uut@placeholdertext{{\large\textmd{PLACEHOLDER --- REPLACE FOR PRINT}}}
\makepagestyle{placeholder}
\makeevenfoot{placeholder}{}{\uut@placeholdertext}{}
\makeoddfoot{placeholder}{}{\uut@placeholdertext}{}
\makeevenhead{placeholder}{}{\uut@placeholdertext}{}
\makeoddhead{placeholder}{}{\uut@placeholdertext}{}
%    \end{macrocode}
%
%
% \begin{environment}{placeholderpage}
%    Define an environment that creates a placeholder page.
%    \begin{macrocode}
\newenvironment{placeholderpage}{
  \begingroup
  \thispagestyle{placeholder}
}{
  \endgroup
  \newpage
}
%    \end{macrocode}
% \end{environment}
%
% \subsection{Dedication Page}
% \begin{environment}{dedication}
%    \begin{macrocode}
\newenvironment{dedication}{
  \begingroup
  \cleartorecto
  \thispagestyle{empty}
  \vspace*{\stretch{3}}%
  \flushright
  \itshape\Large%
}{
  \endflushright
  \vspace*{\stretch{1}}%
  \endgroup
}
%    \end{macrocode}
% \end{environment}
%
% \subsection{Typesetting Paper Lists}
%
% \begin{environment}{paperlist}
% \begin{environment}{paperlist*}
% \begin{macro}{uut@paperitem}
% \begin{macro}{paper}
%   Provide environments to typeset ordered paper lists (|paperlist|)
%   and un-ordered (|paperlist*|) paper lists. These environments are
%   typically used in a list of papers section.
%
%   We use sloppy spacing to avoid major ugliness when setting the
%   bibliography entries.
%
%    \begin{macrocode}
\newcommand{\uut@paperitem}[2]{%
  \midsloppy\item \fullcite{#1}%
  \notblank{#2}{%
    \begingroup%
    \fussy%
    \itshape%
    \smallskip\\%
    #2%
    \endgroup%
  }{}%
}

\newenvironment{paperlist}{
  \begingroup
  \def\paper{\uut@paperitem}
  \enumerate
  \renewcommand{\labelenumi}{\Roman{enumi}.~}
}{%
  \endenumerate
  \endgroup
}

\newenvironment{paperlist*}{
  \begingroup
  \def\paper{\uut@paperitem}
  \itemize
}{%
  \enditemize
  \endgroup
}
%    \end{macrocode}
% \end{macro}
% \end{macro}
% \end{environment}
% \end{environment}
%
%
% \subsection{Typesetting Papers}
% \begin{macro}{\papers}
% Put explanation of |\papers|'s implementation here.
% \begin{macrocode}
\newlength{\uut@paper@rule}
\newcommand{\papers}{%
  \setsecnumdepth{section}
  \renewcommand\thesection{\arabic{section}}
  \renewcommand\thefigure{\arabic{figure}}
  \renewcommand\thetable{\arabic{table}}
  %
  \setlength{\uut@paper@rule}{\paperwidth}
  \addtolength{\uut@paper@rule}{-\textwidth}
  \addtolength{\uut@paper@rule}{-\spinemargin}
  %
}
% \end{macrocode}
% \end{macro}
%
% \begin{macro}{\paper}
%   Typeset a paper front page. We treat papers a bit like chapters
%   and increase the chapter counter. This keeps hyperref happy.
%
%    \begin{macrocode}
\newcommand{\paper}[3]{%
  \cleartorecto%
  \refstepcounter{paper}%
  %
  \thispagestyle{empty}%
  {%
    \parskip\z@%
    \parindent\z@%
    % Heading
    {
      \vspace*{20pt}%
      \flushright%
      \uut@font@paper%
      PAPER%
      \hspace{.4em}%
      \resizebox{!}{25pt}{{\uut@font@paperno\thepaper}}%
      % Shift everything slightly into the right margin to make the
      % paper number look aligned with the title.
      \hspace{-0.07em}%
      \makebox[0pt][l]{%
        \hspace{0.07em}%
        \hspace{.4em}%
        \rule{\uut@paper@rule}{25pt}%
      }%
      \par%
    }%
    \vskip 25pt%
    % Title
    {\uut@font@chapter\HUGE\flushright #1\par}%
    \vskip 60pt%
    % Authors
    {\uut@font@chapter\Large\centering #2\par}%
    \vfill%
    % Copyright
    {\raggedright\noindent\sffamily\footnotesize #3}
  }%
  \addcontentsline{toc}{chapter}{%
    \protect\chapternumberline{\thepaper}#1}%
  \stepcounter{chapter}%
  \newpage
  % Start a new reference section if biblatex is loaded.
  \ifx\newrefsection\uut@undefined
  \else
  \newrefsection%
  \fi
}
%    \end{macrocode}
% \end{macro}
%
% \begin{macro}{\c@paper}
% \begin{macro}{\thepaper}
% \begin{macro}{\papername}
%    \begin{macrocode}
%
% Setup a paper counter so we can reference the paper later.
\newcounter{paper}
\renewcommand{\thepaper}{\Roman{paper}}
\newcommand{\papername}{Paper}
% Monkey patch a Swedish translation
\AtBeginDocument{%
  \ifx \captionsswedish \undefined%
  \else%
  \g@addto@macro\captionsswedish{%
    \def\papername{Artikel}%
  }%
  \fi%
}%
%    \end{macrocode}
% \end{macro}
% \end{macro}
% \end{macro}
%
% \subsection{Miscellaneous Environments and Macros}
%
% \begin{environment}{abstract}
%   Create an abstract environment to consistently typeset abstracts
%   in papers.
%    \begin{macrocode}
\renewenvironment{abstract}{
  \begingroup%
  {\noindent\uut@font@abstract Abstract\,}---%
}{
  \endgroup%
}
%    \end{macrocode}
% \end{environment}
%
% \iffalse
%</class>
% \fi
%
% \subsection{Font Configuration}
%
%    \begin{macrocode}
%<*berling>
\defaultfontfeatures[Berling]{
  Path = fonts/Berling PS/,
  Extension = .pfb,
  UprightFont    = bnr_____,
  BoldFont       = bnb_____,
  ItalicFont     = bni_____,
  BoldItalicFont = bnbi____,
}
%</berling>
%
%<*gill>
\defaultfontfeatures[GillAltOneMT]{
  Path = fonts/Gill PS/,
  Extension = .pfb,
  UprightFont    = gla_____,
  BoldFont       = glab____,
  ItalicFont     = glai____,
  BoldItalicFont = glabi___,
}
%</gill>
%
%<*gillbook>
\defaultfontfeatures[GillAltOneMTBook]{
  Path = fonts/Gill PS/,
  Extension = .pfb,
  UprightFont = glak____,
  ItalicFont  = glaki___,
}
%</gillbook>
%
%<*gillheavy>
\defaultfontfeatures[GillAltOneMTHeavy]{
  Path = fonts/Gill PS/,
  Extension = .pfb,
  UprightFont = glah____,
  ItalicFont  = glahi___,
}
%</gillheavy>
%
%<*gilllight>
\defaultfontfeatures[GillAltOneMTLight]{
  Path = fonts/Gill PS/,
  Extension = .pfb,
  UprightFont = glal____,
  ItalicFont  = glali___,
}
%</gilllight>
%
%    \end{macrocode}
%
% \Finale
\endinput

%% \CharacterTable
%%  {Upper-case    \A\B\C\D\E\F\G\H\I\J\K\L\M\N\O\P\Q\R\S\T\U\V\W\X\Y\Z
%%   Lower-case    \a\b\c\d\e\f\g\h\i\j\k\l\m\n\o\p\q\r\s\t\u\v\w\x\y\z
%%   Digits        \0\1\2\3\4\5\6\7\8\9
%%   Exclamation   \!     Double quote  \"     Hash (number) \#
%%   Dollar        \$     Percent       \%     Ampersand     \&
%%   Acute accent  \'     Left paren    \(     Right paren   \)
%%   Asterisk      \*     Plus          \+     Comma         \,
%%   Minus         \-     Point         \.     Solidus       \/
%%   Colon         \:     Semicolon     \;     Less than     \<
%%   Equals        \=     Greater than  \>     Question mark \?
%%   Commercial at \@     Left bracket  \[     Backslash     \\
%%   Right bracket \]     Circumflex    \^     Underscore    \_
%%   Grave accent  \`     Left brace    \{     Vertical bar  \|
%%   Right brace   \}     Tilde         \~}
%%
